% !TeX TXS-program:compile = txs:///knit2pdf
\documentclass[12pt]{mwart}\usepackage[]{graphicx}\usepackage[]{color}
% maxwidth is the original width if it is less than linewidth
% otherwise use linewidth (to make sure the graphics do not exceed the margin)
\makeatletter
\def\maxwidth{ %
  \ifdim\Gin@nat@width>\linewidth
    \linewidth
  \else
    \Gin@nat@width
  \fi
}
\makeatother

\definecolor{fgcolor}{rgb}{0.345, 0.345, 0.345}
\newcommand{\hlnum}[1]{\textcolor[rgb]{0.686,0.059,0.569}{#1}}%
\newcommand{\hlstr}[1]{\textcolor[rgb]{0.192,0.494,0.8}{#1}}%
\newcommand{\hlcom}[1]{\textcolor[rgb]{0.678,0.584,0.686}{\textit{#1}}}%
\newcommand{\hlopt}[1]{\textcolor[rgb]{0,0,0}{#1}}%
\newcommand{\hlstd}[1]{\textcolor[rgb]{0.345,0.345,0.345}{#1}}%
\newcommand{\hlkwa}[1]{\textcolor[rgb]{0.161,0.373,0.58}{\textbf{#1}}}%
\newcommand{\hlkwb}[1]{\textcolor[rgb]{0.69,0.353,0.396}{#1}}%
\newcommand{\hlkwc}[1]{\textcolor[rgb]{0.333,0.667,0.333}{#1}}%
\newcommand{\hlkwd}[1]{\textcolor[rgb]{0.737,0.353,0.396}{\textbf{#1}}}%
\let\hlipl\hlkwb

\usepackage{framed}
\makeatletter
\newenvironment{kframe}{%
 \def\at@end@of@kframe{}%
 \ifinner\ifhmode%
  \def\at@end@of@kframe{\end{minipage}}%
  \begin{minipage}{\columnwidth}%
 \fi\fi%
 \def\FrameCommand##1{\hskip\@totalleftmargin \hskip-\fboxsep
 \colorbox{shadecolor}{##1}\hskip-\fboxsep
     % There is no \\@totalrightmargin, so:
     \hskip-\linewidth \hskip-\@totalleftmargin \hskip\columnwidth}%
 \MakeFramed {\advance\hsize-\width
   \@totalleftmargin\z@ \linewidth\hsize
   \@setminipage}}%
 {\par\unskip\endMakeFramed%
 \at@end@of@kframe}
\makeatother

\definecolor{shadecolor}{rgb}{.97, .97, .97}
\definecolor{messagecolor}{rgb}{0, 0, 0}
\definecolor{warningcolor}{rgb}{1, 0, 1}
\definecolor{errorcolor}{rgb}{1, 0, 0}
\newenvironment{knitrout}{}{} % an empty environment to be redefined in TeX

\usepackage{alltt}
\usepackage[utf8]{inputenc}
\usepackage[T1,plmath]{polski}
\usepackage{lmodern}

\usepackage{amssymb}
\usepackage{graphicx}
\usepackage{amsmath}
\usepackage{amsthm}
\usepackage[hidelinks]{hyperref}
\usepackage{float}

\title{Sprawozdanie 2}
\author{Aleksander Jakóbczyk i Kacper Pasterniak}
\date{}
\IfFileExists{upquote.sty}{\usepackage{upquote}}{}
\begin{document}
\maketitle



\section*{Lista 5,6 i 7}
\subsection*{Zad 2}
Załóżmy, że 200 losowo wybranych klientów (w różnym wieku) kilku (losowo wybranych) aptek zapytano, jaki lek przeciwbólowy zwykle stosują. Zebrane dane zawarte
są w tablicy 1.

tutaj chyba dodam ta tabelke jeszcze

\begin{kframe}


{\ttfamily\noindent\bfseries\color{errorcolor}{\#\# Error: <text>:3:75: nieoczekiwane ','\\\#\# 2: \ \ \ \ \ \ \ \ dimnames(data) <- list( wyniki\_z\_1 = c("{}18-25"{},"{}26-35"{},"{}36-45"{},"{}46-59"{},"{}60+"{}),\\\#\# 3: \ \ \ \ \ \ \ \ \ \ \ \ \ \ \ \ \ \ \ \ \ \ \ \ \ \ \ \ \ \ \ \ \ \ \ \ \ \ \ \ \ \ \ \ \ \ \ \ \ \ \ \ \ \ \ \ wyniki\_z\_2c = ("{}A"{},\\\#\# \ \ \ \ \ \ \ \ \ \ \ \ \ \ \ \ \ \ \ \ \ \ \ \ \ \ \ \ \ \ \ \ \ \ \ \ \ \ \ \ \ \ \ \ \ \ \ \ \ \ \ \ \ \ \ \ \ \ \ \ \ \ \ \ \ \ \ \ \ \ \ \ \ \ \ \ \ \textasciicircum{}}}\end{kframe}

\subsection*{a)}
Prawdopodobieństwo, że losowo wybrana osoba z badanej populacji w przypadku bólu zażywa Apap jest mniejsze bądź równe 1/4:
\subsection*{b)}
Prawdopodobieństwo, że losowo wybrana osoba z badanej populacji w przypadku bólu zażywa Apap jest równe 1/2:
\subsection*{c)}
Prawdopodobieństwo, że losowo wybrana osoba z badanej populacji w przypadku bólu zażywa Ibuprom jest większe bądź równe 1/5:
\subsection*{d)}
Powtórzyć punkt (a), (b) i (c), ale dla osoby z badanej populacji do lat 35:

\subsection*{Zad 3}
Na podstawie danych w tablicy 1, korzystając z testu Fishera, na poziomie istotności $\alpha = 0.05$, zweryfikować hipotezę, że prawdopodobieństwo, że osoba do lat 35 zażywa Panadol jest równe prawdopodobieństwu, że osoba od 36 lat do 55 lat zażywa Panadol.


Czy na podstawie uzyskanego wyniku można (na zadanym poziomie istotności) odrzucić hipotezę o niezależności wyboru leku Panadol w leczeniu bólu od wieku, przy uwzględnieniu tylko dwóch grup wiekowych - do lat 35 i od 36 do 55 lat?


\subsection*{Zad 4}
Korzystając z funkcji chisq.test w pakiecie R, na poziomie istotności 0.05, zweryfikować hipotezę o niezależności stopnia zadowolenia z pracy i wynagrodzenia na podstawie danych w tablicy 2. Zwrócić uwagę na stosowaną w tej funkcji poprawkę.

tez dodam tabelke


\subsection*{Zad 5}
Napisać deklarację funkcji, która dla danych w tablicy dwudzielczej oblicza wartość poziomu krytycznego (p-value) w asymptotycznym teście niezależności opartym na ilorazie wiarogodności. Korzystając z napisanej funkcji, obliczyć tę wartość dla danych z zadania 4.

\section*{Lista 8 i 9}
\subsection*{Zad 1}
Na podstawie danych z przykładu 1 na wykładzie 7, obliczyć wartości odpowiednich
miar współzmienności zmiennych Segregacja (odpowiedź na pytanie dotyczace segregacji śmieci) i Wiek oraz Segregacja i Miejsce zamieszkania. W przypadku zmiennej
Wiek, wartości miar obliczyć przy “wyjściowych” kategoriach wiekowych, jak również przy połączonych kategoriach wiekowych (jak w przykładzie). Podać interpretację tych wartości. Następnie przeprowadzić analizę korespondencji, tzn. obliczyć wartości odpowiednich macierzy, współrzędnych punktów oraz utworzyć odpowiednie wykresy.

\subsection*{Zad 2}
Załóżmy, że 200 klientów (w różnym wieku) kilku aptek zapytano, jaki lek przeciwbólowy zwykle stosują. Zebrane dane zawarte są w tablicy 2. Na podstawie tych danych, obliczyć odpowiednie miary współzmienności oraz przeprowadzić analizę korespondencji, tzn. obliczyć wartości odpowiednich macierzy, współrzędnych punktów
oraz utworzyć odpowiednie wykresy.

\subsection*{Zad 3}
Na podstawie danych zawartych w tablicy 1, obliczyć (odpowiednią) miarę współzmienności zmiennych Wynagrodzenie i Stopień zadowolenia z pracy (dane te są trochę inne niż te rozpatrywane na poprzednij liście). Następnie, przeprowadzić analizę korespondencji, tzn. obliczyć wartości odpowiednich macierzy, współrzędnych punktów oraz utworzyć odpowiednie wykresy.

\end{document}
